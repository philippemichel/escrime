
\documentclass[french]{scrartcl}
\makeatother
%
\usepackage{geometry} % permet de redéfinir les marges etc. 
\usepackage{graphicx} % Pour incorporer des grapiques, des couleurs etc.
\usepackage{rotating} % Pour  pivoter du texte
\usepackage{array} % Plus facile pour des tableaux complexes
\usepackage{fontspec}
\usepackage[output-decimal-marker={,}]{siunitx} % indisoensable pour les chiffres avec unité
\usepackage{tikz} % Dessis, graphiques etc.
\usepackage{multirow} %Pour fusionner des cellules sur plusieurs lignes dans les tableaux
\usepackage{longtable} % Pour faire des tableaux sur plusieurs pages
\usepackage{textcomp} % Qelques carctèrs supplémentaires
\usepackage{booktabs} % Fugnloe les lignes horizontales dans les tableaux
\usepackage{xspace} % gère les espaces après des fonctions
\usepackage{babel} % Pour écrire en français
\usepackage{hyperref} % Liens hyper texte inernes & externes
%
\hypersetup{% Réglages des liens hypertextes : couleur etc.
colorlinks=true,
pdftitle={},
pdfauthor={Philippe MICHEL},
pdfkeywords={},
unicode
}
%
\setmainfont[Ligatures=TeX]
{MinionPro-Regular}
\setsansfont[Ligatures=TeX]
{TrajanPro}
% Pour avoir une virgule en mode math
\mathcode`\.="013B
%
\newcommand{\pc}[1]{\SI{#1}{\percent}} 
\newcolumntype{C}{>{$}c<{$}} % Colonnes en mode math dans les tableaux


\title{Escrime \& cancer du sein}
\subtitle{Analyse préparatoire}
\author{Philippe MICHEL}
\date{}

\begin{document}
\maketitle

{
\setcounter{tocdepth}{2}
%\tableofcontents
}
\section*{Design de l'étude}\label{design-de-luxe9tude}

Il s'agit de données avant/après, un couple de données par patiente. On
va donc analyser la différence avant/après avec l'hypothèse nulle qu'il
n'y pas de changement.

Le principal reproche qu'on fera à l'étude sera que les patientes, avec
ou sans escrime, auront une évolution de leur état général ressenti.
Quel serait l'évolution du score SF36 chez des femmes suivies pour un
cancer du sein d'un stade/pronostic équivalent \& qui ne participeraient
pas aux séances d'escrime ?

De même, est-ce que les femmes qui participent à ces sénces ne sont pas
un sous-groupe particulier des femmes suivies pour un cancer du sein ?
Moins graves ? Plus motivées ? Moins de douleurs ou autre ?

\section*{Calcul du nombre de cas
nécessaires}

Les données disponibles portent sur 22 cas.

Pour le calcul du nombre de cas nécessaires on va se focaliser sur les
socres globaux (physique \& mental) \& en particulier sur le plus
défavorable, c'est à dire celui pour lequel il y a aura le moins de
différence avant/après.

La moyenne des différences pour le score physique est de 13,12 \& de
13,38 pour le score mental. On fera donc le calcul sur le score
physique. On obtient alors un nombre de patientes nécessaires de 44. Ce
nombre concerne les cas complets c'est à dire avec les données avant \&
après. Prévoir un certain nombre de perdues de vue reste prudent \& une
prévision de 49 cas semble raisonnable.

Comme il y aura des comparaisons multiples sur les différents items il
faudra appliquer une correction qui diminue d'autant la puissance des
tests donc un peu plus de cas serait utile, si possible.

\begin{figure}[h]
  \centering
  \includegraphics[width=.8\linewidth]{figures/phy} 
  \caption{Score "physique"}
  \label{figphy}
\end{figure}

\begin{figure}[h]
  \centering
  \includegraphics[width=.8\linewidth]{figures/ment} 
  \caption{Score "mental"}
  \label{figment}
\end{figure}

\end{document}
